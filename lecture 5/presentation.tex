\documentclass[a4paper, 11pt]{beamer}

\usepackage{polski}
\usepackage[utf8]{inputenc}
\usepackage{units}
\usepackage{wasysym}

\mode<presentation> {
	\usetheme{Frankfurt}
	\setbeamercovered{transparent}
	\usecolortheme{default}
}

\title{Ekonometria Dynamiczna}
\subtitle{Analiza kointegracji}
\author{mgr Paweł Jamer\thanks{pawel.jamer@gmail.com}}

\begin{document}

	\begin{frame}
		\titlepage
	\end{frame}
	
	\section{Integracja}

	\begin{frame}{Integracja}
		\begin{block}{\textbf{Definicja integracji}}
			Szereg czasowy $X_t$ nazwiemy zintegrowanym w stopniu $d$ jeżeli szereg czasowy \[
				\Delta^{d} X_t
			\] jest stacjonarny oraz dla każdego $d^{\prime} < d$ szereg czasowy $\Delta^{d^{\prime}} X_t$ nie jest stacjonarny.
		\end{block}
		\begin{alert}{\textbf{Oznaczenie.}}
			Szereg czasowy $X_t$ zintegrowany w stopniu $d$ oznaczamy symbolem \[
				X_t \sim I\left(d\right).
			\]
		\end{alert}
		\begin{alert}{\textbf{Uwaga.}}
			Stacjonarny szereg czasowy $X_t$ jest zintegrowany w stopniu $0,$ tzn. \[
				X_t \sim I\left(0\right).
			\]
		\end{alert}
	\end{frame}
	
	\begin{frame}{Badanie integracji}
		Integrację szeregów czasowych możemy badać wykorzystując:
		\begin{itemize}
			\item testy DF oraz ADF,
			\item integracyjną statystykę Durbina-Watsona.
		\end{itemize}
	\end{frame}
	
	\begin{frame}{Testy DF oraz ADF}
		W przypadku testów DF oraz ADF hipotezy testowe zapisać możemy w postaci \[
			\begin{cases}
				H_0: & X_t \sim I\left(1\right),\\
				H_1: & X_t \sim I\left(0\right).
			\end{cases}
		\] Tak zapisane hipotezy są de facto najbardziej naturalnym sposobem wyrażenia
		idei na jakiej bazują te testy.
	\end{frame}
	
	\begin{frame}{Integracyjna statystyka Durbina-Watsona}
		\begin{block}{\textbf{Integracyjna statystyka Durbina-Watsona}} \[
			IDW = \frac{\sum_{t=2}^{T}\left(\Delta X_{t}\right)^{2}}{\sum_{t=1}^{T}\left(X_{t} - \overline{X}_t\right)^2}.
		\]
		\end{block}
		\textbf{Interpretacja:}
		\begin{itemize}
			\item wartości statystyki IDW bliskie $0$ sugerują występowanie kointegracji w stopniu $d \geq 1,$
			\item wartości statystyki IDW bliskie $2$ sugerują stacjonarność, tzn.~kointegrację w stopniu $d = 0.$
		\end{itemize}
		\begin{alert}{\textbf{Uwaga.}}
			Możemy przyjąć, że przy poziomie istotności $0.01$ oraz próbie 100 elementowej
			wartość statystyki IDW potrzebna do uzyskania integracji w stopniu $d \geq 1$
			powinna być niższa niż $0.5.$
		\end{alert}
	\end{frame}
	
	\section{Kointegracja}
	
	\begin{frame}{Regresja pozorna}
		\begin{flushleft}
			Kiedy zmienne objaśniające wykazują podobny trend do zmiennej objaśnianej możemy spotkać się z sytuacją regresji pozornej.
		\end{flushleft}
		\begin{flushleft}
			\begin{alert}{\textbf{Przykład.}}
				Związek zachodzący między liczbą narodzin oraz liczbą śmierci.
			\end{alert}
		\end{flushleft}
		\textbf{Problemy:}
		\begin{itemize}
			\item współczynniki regresji mogą być statystycznie istotne,
			\item wartość współczynnika determinacji może być wysoka,
			\item oszacowany model nie opisuje dobrze badanego zjawiska.
		\end{itemize}
	\end{frame}
	
	\begin{frame}{Problem}
		\textbf{Problem:}
		\begin{itemize}
			\item Bazowanie na niestacjonarnych szeregach czasowych podczas 
				budowania modeli może prowadzić do pojawiania się problemu
				regresji pozornej.
			\item Sprowadzając niestacjonarne szeregi czasowe do postaci
				stacjonarnej (poprzez różnicowanie lub eliminację trendu) odbieramy
				sobie możliwość analizowania zależności długookresowych.
		\end{itemize}
		\begin{center}
			\textbf{
				Czy istnieją sytuacje, w których wolno nam bezpiecznie operować na
					zmiennych niestacjonarnych?
			}
		\end{center}
	\end{frame}
	
	\begin{frame}{Definicja}
		\begin{alert}{\textbf{Idea.}}
			Chcemy wiedzieć, czy bazując na pewnej grupie niestacjonarnych 
			szeregów czasowych możemy bezpiecznie zbudować model.
		\end{alert}
		\begin{block}{\textbf{Kointegracja}}
			Powiemy, że szeregi czasowe $X_{1,t}, X_{2,t}, \ldots, X_{n,t}$ są skointegrowane w stopniu $\left(d,b\right),$ jeżeli dla każdego $i=1,2,\ldots,n$ zachodzi \[
				X_{i,t} \sim I\left(d\right)
			\] oraz istnieją takie wartości $\beta_1,\beta_2,\ldots,\beta_n,$ że \[
				\beta_1 X_{1,t} + \beta_2 X_{2,t} + \ldots + \beta_n X_{n,t} \sim I\left(d - b\right).
			\]
		\end{block}
		\begin{alert}{\textbf{Intuicja.}}
			Relacje pomiędzy skointegrowanymi niestacjonarnymi szeregami 
			czasowymi pozostają w długiej perspektywie czasowej niezmienne.
		\end{alert}
	\end{frame}

	\begin{frame}{Testowanie}
		Istnieją dwie powszechnie stosowane metody testowania kointegracji:
		\begin{itemize}
			\item dwustopniowa procedura Engle'a-Grangera -- prostsza, 
				posiadająca liczne ograniczenia,
			\item metoda Johansena -- złożona, dająca dokładne informacje na temat
				kointegracji badanych szeregów czasowych.
		\end{itemize}
	\end{frame}
	
	\begin{frame}{Procedura Engle'a-Grangera}
		\begin{enumerate}
			\item Należy zweryfikować czy wszystkie analizowane szeregi czasowe 
				charakteryzuje ten sam stopień integracji.
			\item Należy zbudować model regresji liniowej wielorakiej w którym 
				jeden z analizowanych szeregów pełni rolę zmiennej objaśnianej, 
				pozosałe natomiast zmiennych objaśniających.
			\item Należy przetestować stopień integracji reszt wyznaczonego w 
				poprzednim kroku modelu.
		\end{enumerate}
	\end{frame}
	
	\section{Model korekty błędem (ECM)}
	
	\begin{frame}{Definicja}
		\begin{block}{\textbf{Model korekty błędem (ECM)}}
			\[
				\Delta y_{t} =
					\mu +
					\alpha \left(y_{t-1} - \beta_{0} - \beta_{1} x_{t-1}\right) + 
					\sum_{i=1}^{k-1} \theta_{i} \Delta y_{t-i} + 
					\sum_{i=0}^{k-1} \gamma_{i} \Delta x_{t-i} + 
					\epsilon_{t}
			\]
		\end{block}
		\textbf{Interpretacja:}
		\begin{itemize}
			\item $y_{t-1} = \beta_{0} + \beta_{1} x_{t-1}$ --- równanie równowagi
				długookresowej,
			\item $y_{t-1} - \beta_{0} - \beta_{1} x_{t-1}$ --- odchylenie od 
				równowagi długookr.,
			\item $\alpha$ --- współczynnik opisujący szybkość dostosowywania się 
				zmiennej objaśnianej do poziomu równowagi długookresowej (w 
				stabilnym modelu $\alpha < 0$).
			\item $\theta_{i}, \gamma_{i}$ --- współczynniki opisujące dynamikę
				krótkookresową.
		\end{itemize}
	\end{frame}
	
	\begin{frame}{Stosowalność}
		\begin{alert}{\textbf{Uwaga.}}
			Twierdzenie Grangera o reprezentacji gwarantuje nam możliwość
			zastosowania mechanizmu korekty błędem względem skointegrowanych
			szeregów czasowych.
		\end{alert}
	\end{frame}
	
	\begin{frame}{Estymacja}
		\begin{enumerate}
			\item Estymacja parametrów równania równowagi długookresowej \[
				y_{t-1} = \beta_{0} + \beta_{1} x_{t-1}.
			\]
			\item Skonstruowanie szeregów czasowych \begin{eqnarray*}
				\epsilon_{t} & = & y_{t} - \beta_{0} - \beta_{1} x_{t},\\
				\Delta x_{t} & = & x_{t} - x_{t-1},\\
				\Delta y_{t} & = & y_{t} - y_{t-1}.\\
			\end{eqnarray*}
			\item Estymacja parametrów równania modelu korekty błędem \[
				\Delta y_{t} =
					\mu +
					\alpha \epsilon_{t-1} + 
					\sum_{i=1}^{k-1} \theta_{i} \Delta y_{t-i} + 
					\sum_{i=0}^{k-1} \gamma_{i} \Delta x_{t-i} + 
					\epsilon_{t}
			\]
		\end{enumerate}
	\end{frame}

	\section*{}

	\begin{frame}
		\center
		\Huge \bfseries
		Pytania?
	\end{frame}

	\begin{frame}
		\center
		\Huge \bfseries
		Dziękuję za uwagę!
	\end{frame}

\end{document}