\documentclass[a4paper, 11pt]{beamer}

\usepackage{polski}
\usepackage[utf8]{inputenc}

\mode<presentation> {
	\usetheme{Frankfurt}
	\setbeamercovered{transparent}
	\usecolortheme{default}
}

\title{Ekonometria Dynamiczna}
\subtitle{Informacje Organizacyjne}
\author{mgr Paweł Jamer\thanks{pawel.jamer@gmail.com}}

\begin{document}

	\begin{frame}
		\titlepage
	\end{frame}

	\begin{frame}{Wymagania wstępne}
		\textbf{Wymagania wstępne:}
		\begin{itemize}
			\item podstawy programowania (r-language),
			\item rachunek prawdopodobieństwa,
			\item statystyka matematyczna.
		\end{itemize}
	\end{frame}
	
	\begin{frame}{Zakres materiału}
		\begin{enumerate}
			\item Szeregi czasowe: podstawowe definicje i miary.
			\item Model liniowy jako narzędzie analizy szeregów czasowych.
			\item Badanie stacjonarności szeregów czasowych.
			\item Wybrane modele jednowymiarowych szeregów czasowych.
			\begin{enumerate}
				\item Model ARMA.
				\item Pochodne modelu ARMA.
				\item Modele z rozkładami opóźnień.
			\end{enumerate}
			\item Analiza kointegracji.
			\item Wybrane modele wielowymiarowych szeregów czasowych.
			\begin{enumerate}
				\item Model VAR.
				\item Model VECM.
			\end{enumerate}
			\item Modelowanie niestacjonarności.
			\begin{enumerate}
				\item Modele ARCH i GARCH.
			\end{enumerate}
		\end{enumerate}
	\end{frame}
	
	\begin{frame}{Zaliczenie (punkty)}
		\begin{enumerate}
			\item \textbf{Teoria} (25 punktów):
			\begin{enumerate}
				\item \textbf{kartkówki} (10 punktów):
				\begin{itemize}
					\item na początku każdych zajęć,
					\item obowiązuje materiał z zajęć poprzednich,
				\end{itemize}
				\item \textbf{egzamin teoretyczny} (15 punktów):
				\begin{itemize}
					\item część egzaminu końcowego,
					\item pytania otwarte z całego materiału teoretycznego.
				\end{itemize}
			\end{enumerate}
			\item \textbf{Praktyka} (75 punktów):
			\begin{enumerate}
				\item \textbf{zadania domowe} (30 punktów):
				\begin{itemize}
					\item podawane na koniec każdych zajęć,
					\item czas na oddanie to średnio 7 dni roboczych,
				\end{itemize}
				\item \textbf{egzamin praktyczny} (45 punktów):
				\begin{itemize}
					\item część egzaminu końcowego,
					\item zadania rozwiązywane na komputerach.
				\end{itemize}
			\end{enumerate}
			\item \textbf{Szeroko pojęta aktywność}:
			\begin{enumerate}
				\item zadania dodatkowe, projekty, prezentacje,
				\item inwencja własna.
			\end{enumerate}
		\end{enumerate}
	\end{frame}
	
	\begin{frame}{Zaliczenie (ocena końcowa)}
		\textbf{Ocena końcowa}:
		\begin{itemize}
			\item \textbf{5.0} $\rightarrow$ od \textbf{91} punktów,
			\item \textbf{4.5} $\rightarrow$ od \textbf{81} do \textbf{90} punktów,
			\item \textbf{4.0} $\rightarrow$ od \textbf{71} do \textbf{80} punktów,
			\item \textbf{3.5} $\rightarrow$ od \textbf{61} do \textbf{70} punktów,
			\item \textbf{3.0} $\rightarrow$ od \textbf{51} do \textbf{60} punktów.
		\end{itemize}
		\textbf{Warunki dodatkowe}:
		\begin{itemize}
			\item minimum 51 punktów z części formalnej (teoria + praktyka).
		\end{itemize}
	\end{frame}
	
	\begin{frame}{Literatura}
		\begin{enumerate}
			\item Iwanik A., Misiewicz J. K.; Wykłady z procesów stochastycznych z zadaniami. Część pierwsza: Procesy Markowa; Script, Warszawa 2010.
			\item Maddala G. S.; Ekonometria; PWN, Warszawa 2008.
			\item Walesiak M., Gatnar E., Statystyczna analiza danych z wykorzystaniem programu R, PWN 2009.
			\item Biecek P., Przewodnik po pakiecie R, Oficyna Wydawnicza GiS, Wrocław 2008.
			\item ...
			\item ...
			\item ...
		\end{enumerate}
	\end{frame}
	
	\begin{frame}{Materiały}
		\Huge\bfseries
		\centering
		http://jamer.webd.pl
	\end{frame}

\end{document}